\documentclass[11pt]{article}
\usepackage[a4paper,margin=1in]{geometry}
\usepackage{amsmath,amssymb}
\usepackage{graphicx}
\usepackage{booktabs}
\usepackage{hyperref}
\usepackage{enumitem}
\usepackage{titlesec}
\titleformat{\section}[block]{\large\bfseries}{}{0pt}{}
\titleformat{\subsection}[block]{\normalsize\bfseries}{}{0pt}{}

\title{Explainer: SpatialAw}
\author{}
\date{}

\begin{document}
\maketitle

\section*{Executive Summary}
SpatialAw uses everyday WiFi signals to tell if someone is present in a room. It does not use cameras or microphones. Instead, it looks at how the WiFi signal changes when a person moves or is present, a measurement called \emph{Channel State Information (CSI)}. We clean the CSI, summarize it into simple motion-related features, and classify each short time window as \textit{occupied} or \textit{vacant}. On the WiAR dataset, our Random Forest model reaches about 92\% accuracy and 0.96 ROC-AUC.

\textbf{Why this is useful}
\begin{itemize}
  \item \textbf{Privacy-first}: No images or audio; just radio signal patterns.
  \item \textbf{Low cost}: Works with commodity WiFi hardware (Intel 5300).
  \item \textbf{Practical}: Real-time on a laptop or small computer.
\end{itemize}

\section*{The Problem}
Detecting human presence is essential for smart buildings, energy management, security, and elder care. Traditional approaches have drawbacks:
\begin{itemize}
  \item \textbf{Cameras}: Privacy concerns; not acceptable in bedrooms/bathrooms.
  \item \textbf{Wearables}: Require user compliance; inconvenient for visitors.
  \item \textbf{PIR sensors}: Limited range; can't see through walls; miss stationary people.
\end{itemize}

WiFi CSI-based detection offers a device-free, privacy-preserving alternative that works through walls and detects even stationary presence by analyzing subtle signal changes.

\section*{What Is CSI?}
WiFi uses many narrow frequency bands (called subcarriers) to send data. CSI tells us how each subcarrier was affected by the room: reflections from walls, furniture, and people. When someone moves, the reflections change slightly, and CSI captures that.

\section*{How It Works (Step by Step)}
\textbf{1) Read CSI}: Each packet carries measurements for 30 subcarriers. We use amplitudes (signal strength) and average across antennas.

\textbf{2) Split into windows}: We cut the stream into short segments (length $T=256$, stride 64). Each window is a small slice of time.

\textbf{3) Clean}: We apply a small moving average to reduce random noise.

\textbf{4) Normalize}: For each subcarrier, we standardize values (z-score) so windows are comparable.

\textbf{5) Summarize with features}: We compute 14 simple metrics that capture motion and variability:
\begin{itemize}
  \item Variability: variance mean/std/max; normalized std mean/std
  \item Envelope: Hilbert envelope mean/std (smooth amplitude)
  \item Motion cues: spectral entropy; velocity mean/max; MAD mean/std
  \item Periodicity: dominant frequency mean/std (from FFT)
\end{itemize}

\textbf{6) Classify}: We train classifiers on these features:
\begin{itemize}
  \item \textbf{Random Forest (RF)}—robust with small datasets, fast, and interpretable.
  \item \textbf{1D-CNN}—learns patterns directly from the window; slightly lower accuracy here.
\end{itemize}

\section*{Data and Labels}
\textbf{Dataset}: We use the WiAR benchmark with 16 activities recorded in a lab. After windowing, we have about 2{,}092 segments.

\textbf{Labels}: We assign \textit{occupied} when motion-related features are strong (e.g., variance + velocity); very low motion becomes \textit{vacant}. We balance the two classes using SMOTE so the model sees enough examples of each.

\section*{Results}
We use group-aware train/test splits to avoid leakage (windows from the same recording are not split across train and test).

\begin{center}
\begin{tabular}{ll}
\toprule
\textbf{Metric} & \textbf{Random Forest} \\
\midrule
Accuracy & 0.92 \\
Precision & 0.91 \\
Recall & 0.93 \\
F1 & 0.92 \\
ROC-AUC & 0.96 \\
\bottomrule
\end{tabular}
\end{center}

\textbf{What this means}: The RF classifier is accurate and fast. High recall (0.93) means it rarely misses an occupied room. Inference is about 0.08 ms per window on a laptop CPU, which is easily real-time.

\section*{Why It Matters}
\begin{itemize}
  \item \textbf{Energy efficiency}: Trigger HVAC/lighting only when occupied—reduce waste in empty rooms.
  \item \textbf{Elder care}: Detect presence/motion without cameras—monitor well-being while respecting privacy.
  \item \textbf{Security}: Spot intrusions or unusual activity patterns—alert when unexpected presence is detected.
  \item \textbf{Space utilization}: Understand room occupancy patterns for better workplace planning.
\end{itemize}

\section*{How to Reproduce (Simple Commands)}
Below is a minimal recipe to run the pipeline. Run these from the repository root.

\begin{verbatim}
# 1) Set up environment
python -m venv .venv
source .venv/bin/activate
pip install -r requirements.txt

# 2) Prepare data (use WiAR or your own Intel 5300 .dat files)
bash scripts/fetch_wiar.sh            # optional helper, if available
python scripts/process_binary_dataset.py --input <path-to-wiar-or-dat>

# 3) Generate windows and features
python scripts/generate_windows.py --window 256 --stride 64
python scripts/extract_features.py --features all

# 4) Train and evaluate models
python model_tools/train_presence_detector.py --model rf
python model_tools/train_presence_detector.py --model cnn

# 5) See outputs
ls models/                            # saved models
ls model_tools/html/                  # metrics, ROC, confusion matrix
\end{verbatim}

\section*{Deploying on the Edge}
\begin{itemize}
  \item \textbf{RF model size}: about 11 MB; runs on CPU.
  \item \textbf{Latency}: \textless1 ms per window typical on laptop-class CPUs.
  \item \textbf{Integration}: Wrap preprocessing + features + RF predict into a loop that reads CSI packets live.
\end{itemize}

\section*{Limitations}
\begin{itemize}
  \item Single lab environment; performance may vary in different rooms.
  \item Binary labels derived from a multi-class dataset; thresholding is simple.
  \item Single-person segments; multiple people not modeled explicitly.
\end{itemize}

\section*{Next Steps}
\begin{itemize}
  \item Train with data from multiple homes/offices to improve generalization.
  \item Learn cross-environment invariants (transfer learning, domain adaptation).
  \item Multi-person counting and activity-aware presence.
  \item Combine hand-crafted features with learned features (hybrid models).
  \item Self-supervised pretraining on large unlabeled CSI streams.
\end{itemize}

\section*{FAQ}
\textbf{Does this record conversations or images?} No. We only use radio signal patterns (CSI), not audio or video.

\textbf{Do I need special hardware?} An Intel 5300-like WiFi NIC that exposes CSI is ideal. For modern hardware, use devices that provide CSI or similar PHY metrics.

\textbf{Will it work in my home?} Likely yes, but accuracy depends on room layout, furniture, and interference. A small calibration dataset helps.

\textbf{Can it count people?} This version detects presence (yes/no). Counting is a planned extension.

\textbf{Is it real-time?} Yes. The RF model predicts in under a millisecond per window on typical CPUs.

\section*{Glossary}
\begin{itemize}
  \item \textbf{CSI (Channel State Information)}: Describes how the WiFi signal traveled through the room.
  \item \textbf{Subcarrier}: A narrow frequency band used in WiFi’s multi-carrier transmission.
  \item \textbf{Window}: A short slice of time over which we summarize the signal.
  \item \textbf{Feature}: A simple numeric summary that helps the classifier (e.g., variance).
  \item \textbf{Random Forest}: An ensemble of decision trees that votes on the final class.
  \item \textbf{ROC-AUC}: A measure of how well the model separates occupied vs. vacant across thresholds.
\end{itemize}

\vspace{8pt}
\noindent\textit{Repo paths:} See \texttt{src/} for loaders and preprocessing; \texttt{model\_tools/} for training/visualization; \texttt{models/} for saved artifacts; \texttt{scripts/} for data preparation.

\end{document}
